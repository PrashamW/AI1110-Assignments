\documentclass[journal,12pt,twocolumn]{IEEEtran}
\usepackage{setspace}
\usepackage{gensymb}
\usepackage{caption}
%\usepackage{multirow}
%\usepackage{multicolumn}
%\usepackage{subcaption}
%\doublespacing
\singlespacing
\usepackage{csvsimple}
\usepackage{amsmath}
\usepackage{multicol}
%\usepackage{enumerate}
\usepackage{amssymb}
%\usepackage{graphicx}
\usepackage{newfloat}
%\usepackage{syntax}
\usepackage{listings}
% \usepackage{iithtlc}
\usepackage{color}
\usepackage{tikz}
\usetikzlibrary{shapes,arrows}



%\usepackage{graphicx}
%\usepackage{amssymb}
%\usepackage{relsize}
%\usepackage[cmex10]{amsmath}
%\usepackage{mathtools}
%\usepackage{amsthm}
%\interdisplaylinepenalty=2500
%\savesymbol{iint}
%\usepackage{txfonts}
%\restoresymbol{TXF}{iint}
%\usepackage{wasysym}
\usepackage{amsthm}
\usepackage{mathrsfs}
\usepackage{txfonts}
\usepackage{stfloats}
\usepackage{cite}
\usepackage{cases}
\usepackage{mathtools}
\usepackage{caption}
\usepackage{enumerate}	
\usepackage{enumitem}
\usepackage{amsmath}
%\usepackage{xtab}
\usepackage{longtable}
\usepackage{multirow}
%\usepackage{algorithm}
%\usepackage{algpseudocode}
\usepackage{enumitem}
\usepackage{mathtools}
\usepackage{hyperref}
%\usepackage[framemethod=tikz]{mdframed}
\usepackage{listings}
    %\usepackage[latin1]{inputenc}                                 %%
    \usepackage{color}                                            %%
    \usepackage{array}                                            %%
    \usepackage{longtable}                                        %%
    \usepackage{calc}                                             %%
    \usepackage{multirow}                                         %%
    \usepackage{hhline}                                           %%
    \usepackage{ifthen}                                           %%
  %optionally (for landscape tables embedded in another document): %%
    \usepackage{lscape}     


\usepackage{url}
\def\UrlBreaks{\do\/\do-}


%\usepackage{stmaryrd}


%\usepackage{wasysym}
%\newcounter{MYtempeqncnt}
\DeclareMathOperator*{\Res}{Res}
%\renewcommand{\baselinestretch}{2}
\renewcommand\thesection{\arabic{section}}
\renewcommand\thesubsection{\thesection.\arabic{subsection}}
\renewcommand\thesubsubsection{\thesubsection.\arabic{subsubsection}}

\renewcommand\thesectiondis{\arabic{section}}
\renewcommand\thesubsectiondis{\thesectiondis.\arabic{subsection}}
\renewcommand\thesubsubsectiondis{\thesubsectiondis.\arabic{subsubsection}}

% correct bad hyphenation here
\hyphenation{op-tical net-works semi-conduc-tor}

%\lstset{
%language=C,
%frame=single, 
%breaklines=true
%}

%\lstset{
	%%basicstyle=\small\ttfamily\bfseries,
	%%numberstyle=\small\ttfamily,
	%language=Octave,
	%backgroundcolor=\color{white},
	%%frame=single,
	%%keywordstyle=\bfseries,
	%%breaklines=true,
	%%showstringspaces=false,
	%%xleftmargin=-10mm,
	%%aboveskip=-1mm,
	%%belowskip=0mm
%}

%\surroundwithmdframed[width=\columnwidth]{lstlisting}
\def\inputGnumericTable{}                                 %%
\lstset{
%language=C,
frame=single, 
breaklines=true,
columns=fullflexible
}
 

\begin{document}
%
\tikzstyle{block} = [rectangle, draw,
    text width=3em, text centered, minimum height=3em]
\tikzstyle{sum} = [draw, circle, node distance=3cm]
\tikzstyle{input} = [coordinate]
\tikzstyle{output} = [coordinate]
\tikzstyle{pinstyle} = [pin edge={to-,thin,black}]

\theoremstyle{definition}
\newtheorem{theorem}{Theorem}[section]
\newtheorem{problem}{Problem}
\newtheorem{proposition}{Proposition}[section]
\newtheorem{lemma}{Lemma}[section]
\newtheorem{corollary}[theorem]{Corollary}
\newtheorem{example}{Example}[section]
\newtheorem{definition}{Definition}[section]
%\newtheorem{algorithm}{Algorithm}[section]
%\newtheorem{cor}{Corollary}
\newcommand{\BEQA}{\begin{eqnarray}}
\newcommand{\EEQA}{\end{eqnarray}}
\newcommand{\define}{\stackrel{\triangle}{=}}
\bibliographystyle{IEEEtran}
%\bibliographystyle{ieeetr}
\providecommand{\nCr}[2]{\,^{#1}C_{#2}} % nCr
\providecommand{\nPr}[2]{\,^{#1}P_{#2}} % nPr
\providecommand{\mbf}{\mathbf}
\providecommand{\pr}[1]{\ensuremath{\Pr\left(#1\right)}}
\providecommand{\qfunc}[1]{\ensuremath{Q\left(#1\right)}}
\providecommand{\sbrak}[1]{\ensuremath{{}\left[#1\right]}}
\providecommand{\lsbrak}[1]{\ensuremath{{}\left[#1\right.}}
\providecommand{\rsbrak}[1]{\ensuremath{{}\left.#1\right]}}
\providecommand{\brak}[1]{\ensuremath{\left(#1\right)}}
\providecommand{\lbrak}[1]{\ensuremath{\left(#1\right.}}
\providecommand{\rbrak}[1]{\ensuremath{\left.#1\right)}}
\providecommand{\cbrak}[1]{\ensuremath{\left\{#1\right\}}}
\providecommand{\lcbrak}[1]{\ensuremath{\left\{#1\right.}}
\providecommand{\rcbrak}[1]{\ensuremath{\left.#1\right\}}}
\theoremstyle{remark}
\newtheorem{rem}{Remark}
% \newcommand{\sgn}{\mathop{\mathrm{sgn}}}
% \providecommand{\abs}[1]{\left\vert#1\right\vert}
% \providecommand{\res}[1]{\Res\displaylimits_{#1}} 
% \providecommand{\norm}[1]{\left\Vert#1\right\Vert}
% \providecommand{\mtx}[1]{\mathbf{#1}}
% \providecommand{\mean}[1]{E\left[ #1 \right]}
\providecommand{\fourier}{\overset{\mathcal{F}}{ \rightleftharpoons}}
%\providecommand{\hilbert}{\overset{\mathcal{H}}{ \rightleftharpoons}}
\providecommand{\system}{\overset{\mathcal{H}}{ \longleftrightarrow}}
	%\newcommand{\solution}[2]{\textbf{Solution:}{#1}}
\newcommand{\solution}{\noindent \textbf{Solution: }}
\newcommand{\myvec}[1]{\ensuremath{\begin{pmatrix}#1\end{pmatrix}}}
\providecommand{\dec}[2]{\ensuremath{\overset{#1}{\underset{#2}{\gtrless}}}}
\DeclarePairedDelimiter{\ceil}{\lceil}{\rceil}
%\numberwithin{equation}{section}
%\numberwithin{problem}{subsection}
%\numberwithin{definition}{subsection}
\makeatletter
\@addtoreset{figure}{section}
\makeatother
\let\StandardTheFigure\thefigure
%\renewcommand{\thefigure}{\theproblem.\arabic{figure}}
\renewcommand{\thefigure}{\thesection}
%\numberwithin{figure}{subsection}
%\numberwithin{equation}{subsection}
%\numberwithin{equation}{section}
%\numberwithin{equation}{problem}
%\numberwithin{problem}{subsection}
\numberwithin{problem}{section}
%%\numberwithin{definition}{subsection}
%\makeatletter
%\@addtoreset{figure}{problem}
%\makeatother
\makeatletter
\@addtoreset{table}{section}
\makeatother
\let\StandardTheFigure\thefigure
\let\StandardTheTable\thetable
\let\vec\mathbf
\numberwithin{equation}{section}
\vspace{3cm}
\title{%Convex Optimization in Python
	\logo{
	Random Numbers
	}
}
%\title{
%	\logo{Matrix Analysis through Octave}{\begin{center}\includegraphics[scale=.24]{tlc}\end{center}}{}{HAMDSP}
%}
% paper title
% can use linebreaks \\ within to get better formatting as desired
%\title{Matrix Analysis through Octave}
%
%
% author names and IEEE memberships
% note positions of commas and nonbreaking spaces ( ~ ) LaTeX will not break
% a structure at a ~ so this keeps an author's name from being broken across
% two lines.
% use \thanks{} to gain access to the first footnote area
% a separate \thanks must be used for each paragraph as LaTeX2e's \thanks
% was not built to handle multiple paragraphs
%
\author{ Prasham Walvekar\\CS21BTECH11047}
% note the % following the last \IEEEmembership and also \thanks - 
% these prevent an unwanted space from occurring between the last author name
% and the end of the author line. i.e., if you had this:
% 
% \author{....lastname \thanks{...} \thanks{...} }
%                     ^------------^------------^----Do not want these spaces!
%
% a space would be appended to the last name and could cause every name on that
% line to be shifted left slightly. This is one of those "LaTeX things". For
% instance, "\textbf{A} \textbf{B}" will typeset as "A B" not "AB". To get
% "AB" then you have to do: "\textbf{A}\textbf{B}"
% \thanks is no different in this regard, so shield the last } of each \thanks
% that ends a line with a % and do not let a space in before the next \thanks.
% Spaces after \IEEEmembership other than the last one are OK (and needed) as
% you are supposed to have spaces between the names. For what it is worth,
% this is a minor point as most people would not even notice if the said evil
% space somehow managed to creep in.
% The paper headers
%\markboth{Journal of \LaTeX\ Class Files,~Vol.~6, No.~1, January~2007}%
%{Shell \MakeLowercase{\textit{et al.}}: Bare Demo of IEEEtran.cls for Journals}
% The only time the second header will appear is for the odd numbered pages
% after the title page when using the twoside option.
% 
% *** Note that you probably will NOT want to include the author's ***
% *** name in the headers of peer review papers.                   ***
% You can use \ifCLASSOPTIONpeerreview for conditional compilation here if
% you desire.
% If you want to put a publisher's ID mark on the page you can do it like
% this:
%\IEEEpubid{0000--0000/00\$00.00~\copyright~2007 IEEE}
% Remember, if you use this you must call \IEEEpubidadjcol in the second
% column for its text to clear the IEEEpubid mark.
% make the title area
\maketitle
\bigskip
\renewcommand{\thefigure}{\theenumi}
\renewcommand{\thetable}{\theenumi}
\begin{abstract}
This is the solution manual of Class Assignment.
\end{abstract}
%%
\section{Unifom Random Numbers}
Let $U$ be a uniform random variable between 0 and 1.
\begin{enumerate}[label=\thesection.\arabic*
,ref=\thesection.\theenumi]
\item Generate $10^6$ samples of $U$ using a C program and save into a file called uni.dat .
\\
\solution
\begin{lstlisting}
wge thttps://github.com/PrashamW/AI1110-Assignments/blob/main/Class_Assignments/Section1/exrand.c
wget https://github.com/PrashamW/AI1110-Assignments/blob/main/Class_Assignments/Section1/coeffs.h
\end{lstlisting}
%
\item
Load the uni.dat file into python and plot the empirical CDF of $U$ using the samples in uni.dat. The CDF is defined as
\begin{align}
F_{U}(x) = \pr{U \le x}
\end{align}
\\
\solution  The following code plots Fig. \ref{fig:cdf_plot(fig1.2)}
\begin{lstlisting}
wget https://github.com/PrashamW/AI1110-Assignments/blob/main/Class_Assignments/Section1/cdf_plot.py
\end{lstlisting}
\begin{figure}
\centering
\includegraphics[width=\columnwidth]{cdf_plot(fig1.2)}
\caption{The CDF of $U$}
\label{fig:cdf_plot(fig1.2)}
\end{figure}
%
\item
Find a  theoretical expression for $F_{U}(x)$.\\
\solution \\
\begin{align}  
f_{U}\brak{x} = 
\begin{cases}
1, & x\in (0,1) \\
0, & \text{otherwise}
\end{cases}
\end{align}
\begin{equation}
    F_U(x) = \int_{-\infty}^{x} f_{U}\brak{x} \,dx  \\
\end{equation}
Hence,
If $ x \leq 0 $,
\begin{align}
    F_U(x) &= \int_{-\infty}^{x} f_{U}\brak{x} \,dx  \\
    F_U(x) &= \int_{-\infty}^{x} 0 \,dx  \\
           &= 0
\end{align}
If $0 <x <1$,
\begin{align}
    F_U(x) &= \int_{0}^{x} f_{U}\brak{x} \,dx  \\
    F_U(x) &= \int_{0}^{x} 1 \,dx  \\
           &= x
\end{align}
If $x \geq 1$,
\begin{align}
    F_U(x) &= \int_{1}^{x} f_{U}\brak{x} \,dx  \\
    F_U(x) &= \int_{1}^{x} 0 \,dx  \\
           &= 0
\end{align}
\item
The mean of $U$ is defined as
%
\begin{equation}
E\sbrak{U} = \frac{1}{N}\sum_{i=1}^{N}U_i
\end{equation}
%
and its variance as
%
\begin{equation}
\text{var}\sbrak{U} = E\sbrak{U- E\sbrak{U}}^2 
\end{equation}
Write a C program to  find the mean and variance of $U$. \\
\solution
\begin{lstlisting}
wget https://github.com/PrashamW/AI1110-Assignments/blob/main/Class_Assignments/Section1/mean_variance.c
wget https://github.com/PrashamW/AI1110-Assignments/blob/main/Class_Assignments/Section1/coeffs.h
\end{lstlisting}

\item Verify your result theoretically given that
\end{enumerate}
%
\begin{equation}
E\sbrak{U^k} = \int_{-\infty}^{\infty}x^kdF_{U}(x)
\end{equation}\solution
\begin{align}  
F_U(x) &= \begin{cases}
x, & x\in (0,1) \\
0, & \text{otherwise}
\end{cases}
\\dF_U(x) &= \begin{cases}
dx, & x\in (0,1) \\
0, & \text{otherwise}
\end{cases}\\
E[U^{k}] &= \int_{-\infty}^{\infty} x^{k}dF_U(x)\\
E[U^{k}] &= \int_{0}^{1} x^{k}dx \\
&= \frac{1}{k+1}
\end{align}
To verify the result obtained for mean and variance from the C code and by the integral:\\
The result given by the C code was,\\Mean = 0.50007 and Variance = 0.083301.\\
Now, using the formula 
\begin{align}
    E[U^{k}] &= \frac{1}{k+1}\\
    E[U] &= \frac{1}{1+1} = \frac{1}{2}\\
    E[U^2] &= \frac{1}{2+1} = \frac{1}{3}
\end{align}
Therefore, Mean = $E[U] = 0.50$\\
Variance = \\
\begin{align}
    E[U^2] - (E[U])^2 &= \frac{1}{3} - \frac{1}{4}\\
    &=\frac{1}{12}\\
    &= 0.0833
\end{align}
Hence the values of mean and variance are almost similar, and is hence theoretically verified.
\section{Central Limit Theorem}
%
\begin{enumerate}[label=\thesection.\arabic*
,ref=\thesection.\theenumi]
%
\item
Generate $10^6$ samples of the random variable
%
\begin{equation}
X = \sum_{i=1}^{12}U_i -6
\end{equation}
%
using a C program, where $U_i, i = 1,2,\dots, 12$ are  a set of independent uniform random variables between 0 and 1
and save in a file called gau.dat\\
\solution
\begin{lstlisting}
wget https://github.com/PrashamW/AI1110-Assignments/blob/main/Class_Assignments/Section2/exrand.c
wget https://github.com/PrashamW/AI1110-Assignments/blob/main/Class_Assignments/Section2/coeffs.h
\end{lstlisting}
%
\item
Load gau.dat in python and plot the empirical CDF of $X$ using the samples in gau.dat. What properties does a CDF have?
\\
\solution The CDF of $X$ is plotted in Fig. \ref{fig:cdf_plot(fig2.2)}\\
\begin{figure}
\centering
\includegraphics[width=\columnwidth]{cdf_plot(fig2.2)}
\caption{The CDF of $X$}
\label{fig:cdf_plot(fig2.2)}
\end{figure}
The CDF of a random variable U has the following properties:
\begin{enumerate}
    \item $F_{U}\brak{x}$ is a non decreasing function of x where $-\infty < x < \infty$ 
    \item $F_{U}\brak{x}$ ranges from 0 to 1 
    \item $F_{U}\brak{x} = 0$ as $x \rightarrow -\infty$ 
    \item $F_{U}\brak{x} = 1$ as $x \rightarrow \infty$
\end{enumerate}
				
\item
Load gau.dat in python and plot the empirical PDF of $X$ using the samples in gau.dat. The PDF of $X$ is defined as
\begin{align}
p_{X}(x) = \frac{d}{dx}F_{X}(x)
\end{align}
What properties does the PDF have?
\\
\solution The PDF of $X$ is plotted in Fig. \ref{fig:pdf_plot(fig2.3)} using the code below
\begin{lstlisting}
wget https://github.com/PrashamW/AI1110-Assignments/blob/main/Class_Assignments/Section2/pdf_plot.py
\end{lstlisting}
The PDF of a random variable X has the following properties:
\begin{enumerate}
    \item The probability density function is non-negative for all the possible values. \\
    \item $ \int_{-\infty}^{\infty} f\brak{x} \,dx  = 1 $ 
    \item $f\brak{x} = 0$ as $x \rightarrow -\infty$
    \item $f\brak{x} = 0$ as $x \rightarrow \infty$
\end{enumerate}
\begin{figure}
\centering
\includegraphics[width=\columnwidth]{pdf_plot(fig2.3)}
\caption{The PDF of $X$}
\label{fig:pdf_plot(fig2.3)}
\end{figure}
\item Find the mean and variance of $X$ by writing a C program.\\
\solution
\begin{lstlisting}
wget https://github.com/PrashamW/AI1110-Assignments/blob/main/Class_Assignments/Section2/mean_variance.c
wget https://github.com/PrashamW/AI1110-Assignments/blob/main/Class_Assignments/Section2/coeffs.h
\end{lstlisting}

\item Given that 
\begin{align}
p_{X}(x) = \frac{1}{\sqrt{2\pi}}\exp\brak{-\frac{x^2}{2}}, -\infty < x < \infty,
\end{align}
repeat the above exercise theoretically.\\
%
\solution
By definition,
\begin{equation}
    p_{X}\brak{x}dx = dF_{U}\brak{x}
\end{equation}
Also, by definition,
\begin{equation}
E\sbrak{U^k} = \int_{-\infty}^{\infty}x^kdF_{U}(x)
\end{equation}
Hence,
\begin{align}
    E\sbrak{U} &= \int_{-\infty}^{\infty} x dF_{U}\brak{x} \,dx \\
    E\sbrak{U} &= \int_{-\infty}^{\infty} x p_{X}\brak{x} \,dx \\
               &= \int_{-\infty}^{\infty} x \frac{1}{\sqrt{2\pi}}exp\brak{-\frac{x^2}{2}} \,dx \\
               &= 0 
\end{align}
Since the function is an odd function its integral over real numbers would be 0
Also,
\begin{align}
    \text{variance} = E\sbrak{U^2} - \brak{E\sbrak{U}}^2\\
    E\sbrak{U^2} =  \int_{-\infty}^{\infty} x^2 p_{X}\brak{x} \,dx \\
                = \int_{-\infty}^{\infty} x^2 \frac{1}{\sqrt{2\pi}}exp\brak{-\frac{x^2}{2}} \,dx \\
                = \int_{-\infty}^{\infty} x x\frac{1}{\sqrt{2\pi}}exp\brak{-\frac{x^2}{2}} \,dx \\
                \intertext{Using integration by parts,} 
\end{align}
\begin{align}
                = \Big|_{-\infty}^{\infty}-x\frac{1}{\sqrt{2\pi}}e^{\brak{-\frac{x^2}{2}}}  +  \int_{-\infty}^{\infty} \frac{1}{\sqrt{2\pi}}exp\brak{-\frac{x^2}{2}} \\
                = 0 + \frac{1}{\sqrt{2\pi}}\sqrt{2\pi} \\
                = 1\\
    \implies \text{variance} = 1
\end{align}
Also,
\begin{align}
   F_{U}\brak{x} &=  \int_{-\infty}^{x} p_{X}\brak{x} \,dx  \\
                 &= \int_{-\infty}^{x} e^{-\frac{x^2}{2}}\,dx 
\end{align}
\end{enumerate}
\section{From Uniform to Other}
\begin{enumerate}[label=\thesection.\arabic*
,ref=\thesection.\theenumi]
%
\item
Generate samples of 
%
\begin{equation}
V = -2\ln\brak{1-U}
\end{equation}
%
and plot its CDF. \\
\solution
\begin{lstlisting}
wget https://github.com/PrashamW/AI1110-Assignments/blob/main/Class_Assignments/Section3/cdf_plot.py
\end{lstlisting}
 The CDF of $X$ is plotted in Fig. \ref{fig:cdf_plot(fig3.1)}\\
\begin{figure}
\centering
\includegraphics[width=\columnwidth]{cdf_plot(fig3.1)}
\caption{The CDF of $X$}
\label{fig:cdf_plot(fig3.1)}
\end{figure}
\item Find a theoretical expression for $F_V(x)$.\\
\solution
\begin{align}
F_{V}(x) &= \pr{V \le x}\\
         &= \pr{-2ln\brak{1-U} \le x}\\
         &= \pr{ln\brak{1-U} \ge -\frac{x}{2}}\\
         &= \pr{1-U \ge e^{-\frac{x}{2}}}\\
         &= \pr{U \le 1-e^{-\frac{x}{2}}}
         \intertext{Let $y = 1-e^{-\frac{x}{2}}$}
         &= \pr{U \le y}\\
         &= F_{U}(y)\\
F_U(y) &= \begin{cases}
y, & y\in (0,1) \\
0, & \text{otherwise}\\
\end{cases}\\
0 &\le 1-e^{-\frac{x}{2}} \le 1\\
0 &\le e^{-\frac{x}{2}} \le 1\\
&\implies x > 0\\
\intertext{Therefore,}
F_{V}(x) = F_{U}(y) &= \begin{cases}
1-e^{-\frac{x}{2}}, & x>0\\
0, & \text{otherwise}
\end{cases}
\end{align}
%
%\item
%Generate the Rayleigh distribution from Uniform. Verify your result through graphical plots.

\end{enumerate}

\section{Triangular Distribution}
\begin{enumerate}[label=\thesection.\arabic*
,ref=\thesection.\theenumi]
%
\item Generate 
	\begin{align}
		T = U_1+U_2
	\end{align}
\solution
\begin{lstlisting}
wget https://github.com/PrashamW/AI1110-Assignments/blob/main/Class_Assignments/Section4/coeffs.h
wget https://github.com/PrashamW/AI1110-Assignments/blob/main/Class_Assignments/Section4/exrand.c
\end{lstlisting}
\item Find the CDF of T.\\
\solution
The CDF of $T$ is plotted in Fig. \ref{fig:cdf_func_plot(fig4.2)}\\
\begin{figure}
\centering
\includegraphics[width=\columnwidth]{cdf_func_plot(fig4.2)}
\caption{The CDF of $T$}
\label{fig:cdf_func_plot(fig4.2)}
\end{figure}
\item Find the PDF of $T$.\\
\solution
The PDF of $T$ is plotted in Fig. \ref{fig:pdf_func_plot(fig4.3)}\\
\begin{figure}
\centering
\includegraphics[width=\columnwidth]{pdf_func_plot(fig4.3)}
\caption{The PDF of $T$}
\label{fig:pdf_func_plot(fig4.3)}
\end{figure}

\item Find the theoretical expressions for the PDF and CDF of $T$.\\
\solution
The CDF of $T$ is given by
	\begin{align}
		F_T(t) = \pr{T \le t} = \pr{U_1 + U_2 \le t}	
	\end{align}		
	Since $U_1, U_2 \in [0,1] \implies U_1 + U_2 \in [0,2]$
	Therefore, if $t \ge 2$, then $U_1 + U_2 \le t$ is always true and if $t < 0$, then $U_1 + U_2 \le t$ is always false.
	
	Now, fix the value of $U_1$ to be some $x$
	\begin{align}
		x + U_2 \le t \implies U_2 \le t - x
	\end{align}
	
	If $0 \le t \le 1$, then $x$ can take all values in $[0,t]$
	\begin{align}
		F_T(t)	&= \int_0^t \pr{U_2 \le t - x} p_{U_1}(x) \, dx \\
		&= \int_0^t F_{U_2}(t-x) p_{U_1}(x) \, dx
	\end{align}
	\begin{align}
		0 \le x \le t &\implies 0 \le t - x \le t \le 1 \\
		&\implies F_{U_2}(t-x) = t - x
	\end{align}
	\begin{align}
		F_T(t) &= \int_0^t (t-x) \cdot 1 \cdot \, dx \\
		&= \left. tx - \frac{x^2}{2} \right|_0^t \\
		&= \frac{t^2}{2}
	\end{align}
	
	If $1 < t < 2$, $x$ can only take values in $[0,1]$ as $U_1 \le 1$
	\begin{align}
		F_T(t)	&= \int_0^1 F_{U_2}(t-x) \cdot 1 \cdot \, dx 
	\end{align}
	\begin{align}
		0 \le x \le t - 1 &\implies 1 \le t - x \le t \\
		t - 1 \le x \le 1 &\implies 0 < t - 1 \le t - x \le 1
	\end{align}
	\begin{align}
		F_T(t) &= \int_0^{t-1} 1 \, dx + \int_{t-1}^1 (t-x)\, dx \\
		&= t - 1 + t(1 - (t - 1)) - \frac{1}{2} + \frac{(t-1)^2}{2} \\
		&= t - 1 + 2t - t^2 -\frac{1}{2} + \frac{t^2}{2} + \frac{1}{2} - t \\ 
		&= -\frac{t^2}{2} + 2t - 1
	\end{align}
	

\item Verify your results through a plot.\\
\solution
The following codes have been used to plot the CDF and PDF
\begin{lstlisting}
wget https://github.com/PrashamW/AI1110-Assignments/blob/main/Class_Assignments/Section4/cdf_plot.py
wget https://github.com/PrashamW/AI1110-Assignments/blob/main/Class_Assignments/Section4/pdf_plot.py
\end{lstlisting}
\begin{figure}
\centering
\includegraphics[width=\columnwidth]{cdf_plot(fig4.5)}
\caption{The CDF of $T$}
\label{fig:cdf_plot(fig4.5)}
\end{figure}
\begin{figure}
\centering
\includegraphics[width=\columnwidth]{pdf_plot(fig4.5)}
\caption{The PDF of $T$}
\label{fig:pdf_plot(fig4.5)}
\end{figure}

\end{enumerate}
\section{Maximum Likelihood}
\begin{enumerate}[label=\thesection.\arabic*
,ref=\thesection.\theenumi]
\item Generate equiprobable $X \in \cbrak{1,-1}$.
\item Generate 
\begin{equation}
Y = AX+N,
\end{equation}
	where $A = 5$ dB,  and $N \sim \gauss{0}{1}$.
	\item Plot $Y$ using a scatter plot.
	\item Guess how to estimate $X$ from $Y$.
\item
\label{ml-ch4_sim}
Find 
\begin{equation}
	P_{e|0} = \pr{\hat{X} = -1|X=1}
\end{equation}
and 
\begin{equation}
	P_{e|1} = \pr{\hat{X} = 1|X=-1}
\end{equation}
%
\item Find $P_e$ assuming that $X$ has equiprobable symbols.
%
\item
Verify by plotting  the theoretical $P_e$ with respect to $A$ from 0 to 10 dB.  
%
\item Now, consider a threshold $\delta$  while estimating $X$ from $Y$. Find the value of $\delta$ that maximizes the theoretical $P_e$.
\item Repeat the above exercise when 
	\begin{align}
		p_{X}(0) = p
	\end{align}
\item Repeat the above exercise using the MAP criterion.
		\end{enumerate}
\section{Gaussian to Other}
\begin{enumerate}[label=\thesection.\arabic*
,ref=\thesection.\theenumi]
\item
Let $X_1 \sim  \gauss{0}{1}$ and $X_2 \sim  \gauss{0}{1}$. Plot the CDF and PDF of
%
\begin{equation}
V = X_1^2 + X_2^2
\end{equation}
%
%
%
\item
If
%
\begin{equation}
F_{V}(x) = 
\begin{cases}
1 - e^{-\alpha x} & x \geq 0 \\
0 & x < 0,
\end{cases}
\end{equation}
%
find $\alpha$.
%
\item
\label{ch3_raleigh_sim}
Plot the CDF and PDf of
%
\begin{equation}
A = \sqrt{V}
\end{equation}
%
\end{enumerate}
\section{Conditional Probability}
\begin{enumerate}[label=\thesection.\arabic*
,ref=\thesection.\theenumi]
\item
\label{ch4_sim}
Plot 
\begin{equation}
P_e = \pr{\hat{X} = -1|X=1}
\end{equation}
%
for 
\begin{equation}
Y = AX+N,
\end{equation}
where $A$ is Raleigh with $E\sbrak{A^2} = \gamma, N \sim \gauss{0}{1}, X \in \brak{-1,1}$ for $0 \le \gamma \le 10$ dB.
%
\item
Assuming that $N$ is a constant, find an expression for $P_e$.  Call this $P_e(N)$
%
\item
%
\label{ch4_anal}
For a function $g$,
\begin{equation}
E\sbrak{g(X)} = \int_{-\infty}^{\infty}g(x)p_{X}(x)\, dx
\end{equation}
%
Find $P_e = E\sbrak{P_e(N)}$.
%
\item
Plot $P_e$ in problems \ref{ch4_sim} and \ref{ch4_anal} on the same graph w.r.t $\gamma$.  Comment.
		\end{enumerate}
\section{Two Dimensions}
Let 
\begin{equation}
\mbf{y} = A\mbf{x} + \mbf{n},
\end{equation}
where 
\begin{align}
x &\in \brak{\mbf{s}_0,\mbf{s}_1}, 
\mbf{s}_0 = 
\begin{pmatrix}
1 
\\
0
\end{pmatrix},
\mbf{s}_1 = 
\begin{pmatrix}
0 
\\
1
\end{pmatrix}
\\
\mbf{n} &= 
\begin{pmatrix}
n_1
\\
n_2
\end{pmatrix},
n_1,n_2 \sim \gauss{0}{1}.
\end{align}
%
\begin{enumerate}[label=\thesection.\arabic*
,ref=\thesection.\theenumi]
%%
\item
\label{ch5_fsk}
Plot 
%
\begin{equation}
\mbf{y}|\mbf{s}_0 \text{ and } \mbf{y}|\mbf{s}_1
\end{equation}
%
on the same graph using a scatter plot.
%
\item
For the above problem, find a decision rule for detecting the symbols $\mbf{s}_0 $ and $\mbf{s}_1$.
%
\item
Plot 
\begin{equation} 
P_e = \pr{\hat{\mbf{x}} = \mbf{s}_1|\mbf{x} = \mbf{s}_0}
\end{equation}
with respect to the SNR from 0 to 10 dB.
%
\item
Obtain an expression for $P_e$. Verify this by comparing the theory and simulation plots on the same graph.
%
\end{enumerate}
\end{document}
