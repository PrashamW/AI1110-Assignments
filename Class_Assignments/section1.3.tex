\documentclass[journal,12pt,two column]{IEEEtran}


\usepackage{listings}
\usepackage{amssymb}
\usepackage[cmex10]{amsmath}
\usepackage{amsthm}
\usepackage[export]{adjustbox}
\usepackage{bm}
\def\inputGnumericTable{} 

\usepackage[latin1]{inputenc}                                 
\usepackage{color}                                            
\usepackage{array}   
\usepackage{longtable}
\usepackage{enumitem}
\usepackage{calc}                                             
\usepackage{multirow}                                         
\usepackage{hhline}                                           
\usepackage{ifthen}  
\usepackage{mathtools}
\usepackage{tikz}
\usepackage{listings}
\usepackage{color}                                            %%
\usepackage{array}                                            %%
\usepackage{caption} 
\usepackage{graphicx}
\graphicspath{{images/}}
\newcommand*{\permcomb}[4][0mu]{{{}^{#3}\mkern#1#2_{#4}}}
\newcommand*{\perm}[1][-3mu]{\permcomb[#1]{P}}\usepackage{setspace}
%\captionsetup[table]{skip=3pt} 

\title{AI1110 Assignment Q-1.3 }
\author{Prasham Walvekar \\ CS21BTECH11047 \\\vspace*{20pt} June 2022}

\begin{document}
\maketitle

\newcommand{\solution}{\noindent \textbf{Solution: }}
\providecommand{\pr}[1]{\ensuremath{\Pr\left(#1\right)}}
\providecommand{\cdf}[2]{\ensuremath{\text{F}_{#1}\left(#2\right)}}
\providecommand{\qfunc}[1]{\ensuremath{Q\left(#1\right)}}
\providecommand{\sbrak}[1]{\ensuremath{{}\left[#1\right]}}
\providecommand{\lsbrak}[1]{\ensuremath{{}\left[#1\right.}}
\providecommand{\rsbrak}[1]{\ensuremath{{}\left.#1\right]}}
\providecommand{\brak}[1]{\ensuremath{\left(#1\right)}}
\providecommand{\lbrak}[1]{\ensuremath{\left(#1\right.}}
\providecommand{\rbrak}[1]{\ensuremath{\left.#1\right)}}
\providecommand{\cbrak}[1]{\ensuremath{\left\{#1\right\}}}
\providecommand{\lcbrak}[1]{\ensuremath{\left\{#1\right.}}
\newcommand*{\comb}[1][-1mu]{\permcomb[#1]{C}}
\renewcommand{\thetable}{\arabic{table}}
\providecommand{\rcbrak}[1]{\ensuremath{\left.#1\right\}}}
\newcommand{\myvec}[1]{\ensuremath{\begin{pmatrix}#1\end{pmatrix}}}
\newcommand{\mydet}[1]{\ensuremath{\begin{vmatrix}#1\end{vmatrix}}}
\let\vec\mathbf

\textbf{1.3:} Find a theoretical expression for $F_U(x)$

\solution

\begin{align}  
f_{U}\brak{x} = 
\begin{cases}
1, & x\in (0,1) \\
0, & \text{otherwise}
\end{cases}
\end{align}
\begin{equation}
    F_U(x) = \int_{-\infty}^{x} f_{U}\brak{x} \,dx  \\
\end{equation}
Hence,
If $ x \leq 0 $,
\begin{align}
    F_U(x) &= \int_{-\infty}^{x} f_{U}\brak{x} \,dx  \\
    F_U(x) &= \int_{-\infty}^{x} 0 \,dx  \\
           &= 0
\end{align}
If $0 <x <1$,
\begin{align}
    F_U(x) &= \int_{0}^{x} f_{U}\brak{x} \,dx  \\
    F_U(x) &= \int_{0}^{x} 1 \,dx  \\
           &= x
\end{align}
If $x \geq 1$,
\begin{align}
    F_U(x) &= \int_{1}^{x} f_{U}\brak{x} \,dx  \\
    F_U(x) &= \int_{1}^{x} 0 \,dx  \\
           &= 0
\end{align}
\end{document}