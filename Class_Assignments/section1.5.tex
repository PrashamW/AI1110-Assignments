\documentclass[journal,12pt,two column]{IEEEtran}


\usepackage{listings}
\usepackage{amssymb}
\usepackage[cmex10]{amsmath}
\usepackage{amsthm}
\usepackage[export]{adjustbox}
\usepackage{bm}
\def\inputGnumericTable{} 

\usepackage[latin1]{inputenc}                                 
\usepackage{color}                                            
\usepackage{array}   
\usepackage{longtable}
\usepackage{enumitem}
\usepackage{calc}                                             
\usepackage{multirow}                                         
\usepackage{hhline}                                           
\usepackage{ifthen}  
\usepackage{mathtools}
\usepackage{tikz}
\usepackage{listings}
\usepackage{color}                                            %%
\usepackage{array}                                            %%
\usepackage{caption} 
\usepackage{graphicx}
\graphicspath{{images/}}
\newcommand*{\permcomb}[4][0mu]{{{}^{#3}\mkern#1#2_{#4}}}
\newcommand*{\perm}[1][-3mu]{\permcomb[#1]{P}}\usepackage{setspace}
%\captionsetup[table]{skip=3pt} 

\title{AI1110 Assignment Q-1.5 }
\author{Prasham Walvekar \\ CS21BTECH11047 \\\vspace*{20pt} June 2022}

\begin{document}
\maketitle

\newcommand{\solution}{\noindent \textbf{Solution: }}
\providecommand{\pr}[1]{\ensuremath{\Pr\left(#1\right)}}
\providecommand{\cdf}[2]{\ensuremath{\text{F}_{#1}\left(#2\right)}}
\providecommand{\qfunc}[1]{\ensuremath{Q\left(#1\right)}}
\providecommand{\sbrak}[1]{\ensuremath{{}\left[#1\right]}}
\providecommand{\lsbrak}[1]{\ensuremath{{}\left[#1\right.}}
\providecommand{\rsbrak}[1]{\ensuremath{{}\left.#1\right]}}
\providecommand{\brak}[1]{\ensuremath{\left(#1\right)}}
\providecommand{\lbrak}[1]{\ensuremath{\left(#1\right.}}
\providecommand{\rbrak}[1]{\ensuremath{\left.#1\right)}}
\providecommand{\cbrak}[1]{\ensuremath{\left\{#1\right\}}}
\providecommand{\lcbrak}[1]{\ensuremath{\left\{#1\right.}}
\newcommand*{\comb}[1][-1mu]{\permcomb[#1]{C}}
\renewcommand{\thetable}{\arabic{table}}
\providecommand{\rcbrak}[1]{\ensuremath{\left.#1\right\}}}
\newcommand{\myvec}[1]{\ensuremath{\begin{pmatrix}#1\end{pmatrix}}}
\newcommand{\mydet}[1]{\ensuremath{\begin{vmatrix}#1\end{vmatrix}}}
\let\vec\mathbf

\textbf{1.5:} Verify your result theoretically given that $E[U^{k}] = \int_{-\infty}^{\infty} x^{k}dF_U(x)$

\solution

\begin{align}  
F_U(x) = \begin{cases}
x, & x\in (0,1) \\
0, & \text{otherwise}
\end{cases}
\\dF_U(x) = \begin{cases}
dx, & x\in (0,1) \\
0, & \text{otherwise}
\end{cases}
\end{align}
\begin{align} 
E[U^{k}] = \int_{-\infty}^{\infty} x^{k}dF_U(x)
\end{align}
\begin{equation}
    E[U^{k}] = \int_{0}^{1} x^{k}dx \\
\end{equation}
\begin{equation}
    E[U^{k}] = \frac{1}{k+1}
\end{equation}
To verify the result obtained for mean and variance from the C code and by the integral:\\
The result given by the C code was,\\Mean = 0.50007 and Variance = 0.083301.\\
Now, using the formula 
\begin{align}
    E[U^{k}] = \frac{1}{k+1}\\
    E[U] = \frac{1}{1+1} = \frac{1}{2}\\
    E[U^2] = \frac{1}{2+1} = \frac{1}{3}\\
\end{align}
Therefore, Mean = $E[U] = 0.5$\\
Variance = \\
\begin{align}
    E[U^2] - (E[U])^2 = \frac{1}{3} - \frac{1}{4}\\
    =\frac{1}{12} = 0.0833
\end{align}
Hence the values of mean and variance are almost similar, and is hence theoretically verified.
\end{document}